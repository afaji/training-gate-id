 \documentclass{beamer}
\usetheme{tokitex}

\usepackage{tikz}
\usepackage{graphics}
\usepackage{multirow}
\usepackage{tabto}
\usepackage{xspace}
\usepackage{amsmath}

\usepackage{tikz}
\usepackage{clrscode3e}

\usepackage[english,bahasa]{babel}
\newtranslation[to=bahasa]{Section}{Bagian}
\newtranslation[to=bahasa]{Subsection}{Subbagian}

\usepackage{listings, lstautogobble}
\usepackage{color}

\definecolor{dkgreen}{rgb}{0,0.6,0}
\definecolor{gray}{rgb}{0.5,0.5,0.5}
\definecolor{mauve}{rgb}{0.58,0,0.82}

\lstset{frame=tb,
  language=pascal,
  aboveskip=1mm,
  belowskip=1mm,
  showstringspaces=false,
  columns=fullflexible,
  keepspaces=true,
  basicstyle={\small\ttfamily},
  numbers=none,
  numberstyle=\tiny\color{gray},
  keywordstyle=\color{blue},
  commentstyle=\color{dkgreen},
  stringstyle=\color{mauve},
  breaklines=true,
  breakatwhitespace=true,
  autogobble=true
}

\usepackage{caption}
\captionsetup[figure]{labelformat=empty}

\newcommand{\progTerm}[1]{\textbf{#1}}
\newcommand{\foreignTerm}[1]{\textit{#1}}
\newcommand{\newTerm}[1]{\alert{\textbf{#1}}}
\newcommand{\emp}[1]{\alert{#1}}
\newcommand{\statement}[1]{"#1"}

% Getting tired of writing \foreignTerm all the time
\newcommand{\farray}{\foreignTerm{array}\xspace}
\newcommand{\fArray}{\foreignTerm{Array}\xspace}
\newcommand{\foverhead}{\foreignTerm{overhead}\xspace}
\newcommand{\fOverhead}{\foreignTerm{Overhead}\xspace}
\newcommand{\fsubarray}{\foreignTerm{subarray}\xspace}
\newcommand{\fSubarray}{\foreignTerm{Subarray}\xspace}
\newcommand{\fbasecase}{\foreignTerm{base case}\xspace}
\newcommand{\fBasecase}{\foreignTerm{Base case}\xspace}
\newcommand{\ftopdown}{\foreignTerm{top down}\xspace}
\newcommand{\fTopdown}{\foreignTerm{Top down}\xspace}
\newcommand{\fbottomup}{\foreignTerm{bottom up}\xspace}
\newcommand{\fBottomup}{\foreignTerm{Bottom up}\xspace}
\newcommand{\fpruning}{\foreignTerm{pruning}\xspace}
\newcommand{\fPruning}{\foreignTerm{Pruning}\xspace}

\newcommand{\fgraph}{\foreignTerm{graph}\xspace}
\newcommand{\fGraph}{\foreignTerm{Graph}\xspace}
\newcommand{\fnode}{\foreignTerm{node}\xspace}
\newcommand{\fNode}{\foreignTerm{Node}\xspace}
\newcommand{\fedge}{\foreignTerm{edge}\xspace}
\newcommand{\fEdge}{\foreignTerm{Edge}\xspace}
\newcommand{\fdegree}{\foreignTerm{degree}\xspace}
\newcommand{\fDegree}{\foreignTerm{Degree}\xspace}
\newcommand{\fadjacencylist}{\foreignTerm{adjacency list}\xspace}
\newcommand{\fAdjacencylist}{\foreignTerm{Adjacency list}\xspace}
\newcommand{\fadjacencymatrix}{\foreignTerm{adjacency matrix}\xspace}
\newcommand{\fAdjacencymatrix}{\foreignTerm{Adjacency matrix}\xspace}
\newcommand{\fedgelist}{\foreignTerm{edge list}\xspace}
\newcommand{\fEdgelist}{\foreignTerm{Edge list}\xspace}
\newcommand{\flist}{\foreignTerm{list}\xspace}
\newcommand{\fList}{\foreignTerm{List}\xspace}
\newcommand{\fgraphtraversal}{\foreignTerm{graph traversal}\xspace}
\newcommand{\fGraphtraversal}{\foreignTerm{Graph traversal}\xspace}
\newcommand{\ftree}{\foreignTerm{tree}\xspace}
\newcommand{\fTree}{\foreignTerm{Tree}\xspace}

\newcommand{\fDivideAndConquer}{\foreignTerm{Divide and Conquer}\xspace}
\newcommand{\fMergeSort}{\foreignTerm{Merge Sort}\xspace}
\newcommand{\fQuickSort}{\foreignTerm{Quicksort}\xspace}
\newcommand{\fpivot}{\foreignTerm{pivot}\xspace}
\newcommand{\fPivot}{\foreignTerm{Pivot}\xspace}
\newcommand{\fBruteForce}{\foreignTerm{Brute Force}\xspace}
\newcommand{\fCompleteSearch}{\foreignTerm{Complete Search}\xspace}
\newcommand{\fExhaustiveSearch}{\foreignTerm{Exhaustive Search}\xspace}
\newcommand{\fBinarySearch}{\foreignTerm{Binary Search}\xspace}
\newcommand{\fGreedy}{\foreignTerm{Greedy}\xspace}
\newcommand{\fGreedyChoice}{\foreignTerm{Greedy Choice}\xspace}

\newcommand{\pheap}{\progTerm{heap}\xspace}
\newcommand{\pHeap}{\progTerm{Heap}\xspace}
\newcommand{\pBinaryHeap}{\progTerm{Binary Heap}\xspace}
\newcommand{\pbinaryHeap}{\progTerm{binary heap}\xspace}

\title{Memenangkan Kompetisi}
\author{Tim Olimpiade Komputer Indonesia}
\date{}

\begin{document}

\begin{frame}
  \titlepage
\end{frame}

  
\begin{frame}
  \frametitle{Kompetisi}
  Kompetisi pemrograman seperti ACM-ICPC, IOI, dan tentunya OSN membutuhkan persiapan yang matang.
\end{frame}

\begin{frame}
  \frametitle{Mengenali Medan}
  Mengenali kondisi kompetisi yang Anda ikuti adalah hal yang penting. Kondisi ini antara lain:
  \begin{itemize}
    \item Perhitungan poin/cara pemilihan pemenang.
    \item Jenis soal.
    \item Jumlah dan tingkat kesulitan soal.
    \item Waktu yang diberikan.
    \item \foreignTerm{Resource} yang ada.
  \end{itemize}
\end{frame}


\begin{frame}
  \frametitle{Perhitungan Poin}
  Ada dua jenis cara perhitungan poin yang umum digunakan pada kompetisi pemrograman:
  \begin{itemize}
    \item IOI \foreignTerm{style}\newline
     Anda bisa mendapatkan poin secara parsial, tergantung \foreignTerm{subtask} yang and kerjakan. Waktu pengerjaan tidak berpengaruh pada hasil akhir. IOI dan OSN menggunakan sistem penilaian ini.
    
    \item ACM \foreignTerm{style}\newline
    Anda harus menyelesaikan masalah secara keseluruhan. Dengan kata lain, nilai Anda adalah 0 atau 100. Penentuan pemenang dihitung dari total soal yang diselesaikan.  Waktu pengerjaan dihitung sebagai penalti.
  \end{itemize}
\end{frame}

\begin{frame}
  \frametitle{Jenis Soal}
  \begin{itemize}
    \item Soal yang diberikan pada tingkat kompetisi umumnya membutuhkan analisis yang dalam dan berlapis, khususnya untuk IOI \foreignTerm{style}.
    \item Tidak mungkin Anda temukan soal: diberikan \farray, cetak elemen-elemen dalam \farray itu secara terurut!
    \item Mengetahui teori-teori dan menghafal algoritma saja tidak cukup.
    \item Yang diuji adalah kemampuan Anda melakukan observasi dan analisis, untuk menyelesaikan masalah.
  \end{itemize}
\end{frame}

\begin{frame}
  \frametitle{Jumlah dan Tingkat Kesulitan Soal}
    
  \begin{itemize}
    \item OSN terdiri dari 3 sampai 4 soal perhari selama dua hari
    \item Kesulitan soal OSN bervariasi mulai dari mudah sampai sulit.
    \item Biasanya akan ada soal yang bertipe \foreignTerm{open problem}. Soal ini tidak memiliki solusi pasti. Artinya Anda akan membuat program yang mendekati kebenaran sebaik mungkin.
  \end{itemize}
\end{frame}

\begin{frame}
  \frametitle{Waktu dan \textit{Resource}}
  \begin{itemize}
    \item Waktu kontes OSN adalah 5 jam perhari.
    \item Perhatikan \foreignTerm{resouce} berupa:
    \begin{itemize}
      \item \foreignTerm{Compiler} yang disediakan.
      \item \foreignTerm{Text editor} yang disediakan.
      \item Sistem operasi yang disediakan.
      \item Kemampuan mesin.
    \end{itemize}
    \item Biasakan diri dengan \foreignTerm{sistem operasi}, \foreignTerm{editor} serta \foreignTerm{compiler} yang akan dipakai
    \item Informasi kemampuan mesin dapat digunakan untuk memprediksi \foreignTerm{running time}
  \end{itemize}
\end{frame}

\begin{frame}
  \frametitle{Persiapan Sebelum Kompetisi}
  \begin{itemize}
    \item   Lakukan latihan secara berkala sebelum kompetisi. Latihan bisa dilakukan di:
    \begin{itemize}
      \item TLX
      \item Codeforces
      \item Topcoder
      \item SPOJ
      \item COCI
      \item dll
    \end{itemize}
    \item Simulasikan OSN. Bisa menyelesaikan soal setingkat OSN saja tidak cukup.
    \item Anda harus dapat menyelesaikannya dalam situasi yang serupa saat kompetisi. Buatlah simulasi OSN dengan memilih 3 buah soal, kemudian kerjakan sebaik-baiknya selama 5 jam tanpa bantuan.
  \end{itemize}
\end{frame}

\begin{frame}
  \frametitle{Tips Sebelum Kompetisi}
  \begin{itemize}
    \item Istirahat cukup sebelum kontes berlangsung. Tidak dianjurkan untuk begadang.
    \item Jika Anda sudah melakukan persiapan yang matang jauh-jauh hari sebelumnya, beristirahatlah pada H-1 pertandingan agar pikiran Anda tenang.
  \end{itemize}
  
\end{frame}


\begin{frame}
  \frametitle{Tips Saat Kompetisi (lanj.)}
  \begin{itemize}
    \item Disarankan untuk membaca semua soal terlebih dahulu sebelum memulai mengerjakan
    \item Pastikan Anda sudah membaca semua soal.
    \item Jangan sampai ada penyesalan di akhir karena kehabisan waktu untuk mengerjakan soal lain, sementara ada soal lainnya yang ternyata bisa Anda kerjakan, luput dari perhatian Anda.
  \end{itemize}
  
\end{frame}

\begin{frame}
  \frametitle{Tips Saat Kompetisi (lanj.)}
  \begin{itemize}
    \item Ketika menemukan suatu algoritma untuk menyelesaikan soal, luangkan waktu beberapa menit untuk memikirkan kembali algoritma tersebut matang-matang. 
    \item Pastikan algoritma ini bekerja untuk soal yang akan Anda selesaikan. 
    \item Hal ini untuk menghindari membuang-buang waktu karena Anda baru sadar ada kesalahan algoritma saat Anda sudah mulai menulis program.
  \end{itemize}
\end{frame}


\begin{frame}
  \frametitle{Tips Saat Kompetisi (lanj.)}
  \begin{itemize}
    \item Lima jam adalah waktu yang panjang. Beristirahatlah sejenak jika merasa lelah, karena ketika lelah, konsentrasi akan berkurang
    \item Atur makanan Anda saat dan sebelum kontes. Jangan sampai terlalu lapar atau terlalu kenyang.
    \item Sangat dianjurkan untuk mengerjakan semua soal, meskipun tidak mendapatkan nilai sempurna. Anda bisa mendapatkan beberapa poin ekstra dengan mengerjakan \foreignTerm{subtask} yang diberikan.
  \end{itemize}
\end{frame}

\begin{frame}
  \frametitle{Penutup}
  \begin{itemize}
    \item Anda telah mempelajari dasar pemrograman kompetitif.
    \item Teruslah berlatih dan mempelajari hal baru untuk meningkatkan kemampuan Anda.
    \item Bila Anda akan menghadapi kompetisi: selamat berkompetisi!\newline
    Semoga sukses dan jangan lupa untuk bersenang-senang :)
  \end{itemize}
\end{frame}

\end{document}
