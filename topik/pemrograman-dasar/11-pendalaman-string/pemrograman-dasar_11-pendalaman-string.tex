\documentclass{beamer}
\usetheme{tokitex}

\usepackage{tikz}
\usepackage{graphics}
\usepackage{multirow}
\usepackage{tabto}
\usepackage{xspace}
\usepackage{amsmath}

\usepackage{tikz}
\usepackage{clrscode3e}

\usepackage[english,bahasa]{babel}
\newtranslation[to=bahasa]{Section}{Bagian}
\newtranslation[to=bahasa]{Subsection}{Subbagian}

\usepackage{listings, lstautogobble}
\usepackage{color}

\definecolor{dkgreen}{rgb}{0,0.6,0}
\definecolor{gray}{rgb}{0.5,0.5,0.5}
\definecolor{mauve}{rgb}{0.58,0,0.82}

\lstset{frame=tb,
  language=pascal,
  aboveskip=1mm,
  belowskip=1mm,
  showstringspaces=false,
  columns=fullflexible,
  keepspaces=true,
  basicstyle={\small\ttfamily},
  numbers=none,
  numberstyle=\tiny\color{gray},
  keywordstyle=\color{blue},
  commentstyle=\color{dkgreen},
  stringstyle=\color{mauve},
  breaklines=true,
  breakatwhitespace=true,
  autogobble=true
}

\usepackage{caption}
\captionsetup[figure]{labelformat=empty}

\newcommand{\progTerm}[1]{\textbf{#1}}
\newcommand{\foreignTerm}[1]{\textit{#1}}
\newcommand{\newTerm}[1]{\alert{\textbf{#1}}}
\newcommand{\emp}[1]{\alert{#1}}
\newcommand{\statement}[1]{"#1"}

% Getting tired of writing \foreignTerm all the time
\newcommand{\farray}{\foreignTerm{array}\xspace}
\newcommand{\fArray}{\foreignTerm{Array}\xspace}
\newcommand{\foverhead}{\foreignTerm{overhead}\xspace}
\newcommand{\fOverhead}{\foreignTerm{Overhead}\xspace}
\newcommand{\fsubarray}{\foreignTerm{subarray}\xspace}
\newcommand{\fSubarray}{\foreignTerm{Subarray}\xspace}
\newcommand{\fbasecase}{\foreignTerm{base case}\xspace}
\newcommand{\fBasecase}{\foreignTerm{Base case}\xspace}
\newcommand{\ftopdown}{\foreignTerm{top down}\xspace}
\newcommand{\fTopdown}{\foreignTerm{Top down}\xspace}
\newcommand{\fbottomup}{\foreignTerm{bottom up}\xspace}
\newcommand{\fBottomup}{\foreignTerm{Bottom up}\xspace}
\newcommand{\fpruning}{\foreignTerm{pruning}\xspace}
\newcommand{\fPruning}{\foreignTerm{Pruning}\xspace}

\newcommand{\fgraph}{\foreignTerm{graph}\xspace}
\newcommand{\fGraph}{\foreignTerm{Graph}\xspace}
\newcommand{\fnode}{\foreignTerm{node}\xspace}
\newcommand{\fNode}{\foreignTerm{Node}\xspace}
\newcommand{\fedge}{\foreignTerm{edge}\xspace}
\newcommand{\fEdge}{\foreignTerm{Edge}\xspace}
\newcommand{\fdegree}{\foreignTerm{degree}\xspace}
\newcommand{\fDegree}{\foreignTerm{Degree}\xspace}
\newcommand{\fadjacencylist}{\foreignTerm{adjacency list}\xspace}
\newcommand{\fAdjacencylist}{\foreignTerm{Adjacency list}\xspace}
\newcommand{\fadjacencymatrix}{\foreignTerm{adjacency matrix}\xspace}
\newcommand{\fAdjacencymatrix}{\foreignTerm{Adjacency matrix}\xspace}
\newcommand{\fedgelist}{\foreignTerm{edge list}\xspace}
\newcommand{\fEdgelist}{\foreignTerm{Edge list}\xspace}
\newcommand{\flist}{\foreignTerm{list}\xspace}
\newcommand{\fList}{\foreignTerm{List}\xspace}
\newcommand{\fgraphtraversal}{\foreignTerm{graph traversal}\xspace}
\newcommand{\fGraphtraversal}{\foreignTerm{Graph traversal}\xspace}
\newcommand{\ftree}{\foreignTerm{tree}\xspace}
\newcommand{\fTree}{\foreignTerm{Tree}\xspace}

\newcommand{\fDivideAndConquer}{\foreignTerm{Divide and Conquer}\xspace}
\newcommand{\fMergeSort}{\foreignTerm{Merge Sort}\xspace}
\newcommand{\fQuickSort}{\foreignTerm{Quicksort}\xspace}
\newcommand{\fpivot}{\foreignTerm{pivot}\xspace}
\newcommand{\fPivot}{\foreignTerm{Pivot}\xspace}
\newcommand{\fBruteForce}{\foreignTerm{Brute Force}\xspace}
\newcommand{\fCompleteSearch}{\foreignTerm{Complete Search}\xspace}
\newcommand{\fExhaustiveSearch}{\foreignTerm{Exhaustive Search}\xspace}
\newcommand{\fBinarySearch}{\foreignTerm{Binary Search}\xspace}
\newcommand{\fGreedy}{\foreignTerm{Greedy}\xspace}
\newcommand{\fGreedyChoice}{\foreignTerm{Greedy Choice}\xspace}

\newcommand{\pheap}{\progTerm{heap}\xspace}
\newcommand{\pHeap}{\progTerm{Heap}\xspace}
\newcommand{\pBinaryHeap}{\progTerm{Binary Heap}\xspace}
\newcommand{\pbinaryHeap}{\progTerm{binary heap}\xspace}

\title{Pendalaman String\newline (Pascal)}
\author{Tim Olimpiade Komputer Indonesia}
\date{}

\begin{document}

\begin{frame}
\titlepage
\end{frame}

\begin{frame}
\frametitle{Pendahuluan}
Melalui dokumen ini, kalian akan:
\begin{itemize}
  \item Memahami lebih dalam mengenai string.
  \item Mengenal fungsi-fungsi dasar untuk pengolahan string.
\end{itemize}
\vfill
Seluruh pembahasan string pada dokumen ini mengacu pada tipe data \textbf{string} pada \textbf{Pascal}.
\end{frame}

\section{Pengolahan String}
\frame{\sectionpage}

\begin{frame}
\frametitle{Pengolahan String}
Pascal menyediakan berbagai fungsi dan prosedur dasar pengolahan string, beberapa di antaranya adalah:
\begin{itemize}
  \item length
  \item pos
  \item copy
  \item delete
  \item insert
  \item str
  \item val
\end{itemize}

Sebagai catatan, pada Pascal, indeks atau posisi dari karakter pertama pada sebuah string adalah 1 (\textit{1-based indexing}).
\end{frame}

\begin{frame}[fragile]
\frametitle{Pengolahan String:\newline length}
\begin{block}{length(S)}
Merupakan fungsi yang mengembalikan panjang dari string $S$.
\end{block}

Contoh:
\begin{lstlisting}
  s1 := 'bebek';
  s2 := '';
  s3 := 'Pak Dengklek';

  writeln(length(s1)); (* mencetak 5 *)
  writeln(length(s2)); (* mencetak 0 *)
  writeln(length(s3)); (* mencetak 12 *)
\end{lstlisting}
\end{frame}

\begin{frame}[fragile]
\frametitle{Pengolahan String:\newline pos}
\begin{block}{pos(T, S)}
Merupakan fungsi yang mencari dan mengembalikan posisi terawal substring $T$ dari suatu string $S$.

Jika tidak ditemukan, dikembalikan nilai 0.
\end{block}
Contoh:
\begin{lstlisting}
  s := 'Pak Dengklek berternak';
  t1 := 'Dengklek';
  t2 := 'pak';
  t3 := 'klek';

  writeln(pos(t1, s)); (* mencetak 4 *)
  writeln(pos(t2, s)); (* mencetak 0, tidak ditemukan  *)
  writeln(pos(t3, s)); (* mencetak 9 *)
\end{lstlisting}
\end{frame}

\begin{frame}[fragile]
\frametitle{Pengolahan String:\newline copy}
\begin{block}{copy(S, pos, cnt)}
Merupakan fungsi yang mengembalikan substring dari indeks $pos$ sebanyak $cnt$ karakter dari string $S$.
\end{block}
Contoh:
\begin{lstlisting}
  s := 'Pak Dengklek berternak';

  writeln(copy(s, 1, 6)); (* mencetak 'Pak De' *)
  writeln(copy(s, 3, 1)); (* mencetak 'k' *)
\end{lstlisting}
\end{frame}

\begin{frame}[fragile]
\frametitle{Pengolahan String:\newline delete}
\begin{block}{delete(S, pos, cnt)}
Merupakan prosedur yang menghapus substring dari indeks $pos$ sebanyak $cnt$ karakter dari string $S$.

Parameter $S$ dipanggil dengan \textit{by reference}.
\end{block}
Contoh:
\begin{lstlisting}
  s := 'Pak Dengklek berternak';

  delete(s, 2, 3);
  writeln(s); (* mencetak 'PDengklek berternak' *)
\end{lstlisting}
\end{frame}

\begin{frame}[fragile]
\frametitle{Pengolahan String:\newline insert}
\begin{block}{insert(T, S, pos)}
Merupakan prosedur yang menyisipkan string $T$ ke dalam string $S$ mulai dari indeks $pos$.

Parameter $S$ dipanggil dengan \textit{by reference}.
\end{block}
Contoh:
\begin{lstlisting}
  s := 'Pak Dengklek berternak';
  t := 'dan Bu ';

  insert(t, s, 5);
  writeln(s); (* mencetak 'Pak dan Bu Dengklek berternak' *)
\end{lstlisting}
\end{frame}

\begin{frame}[fragile]
\frametitle{Pengolahan String:\newline str}
\begin{block}{str(v, S)}
Merupakan prosedur yang mengkonversi suatu data numerik $v$ menjadi string, dan ditampung ke dalam string $S$.

Parameter $S$ dipanggil dengan \textit{by reference}.
\end{block}
Contoh:
\begin{lstlisting}
  s := '';
  nilai := 781;

  str(nilai, s);
  writeln(s);         (* mencetak '781' *)
  writeln(length(s)); (* mencetak 3 *)
\end{lstlisting}
\end{frame}

\begin{frame}[fragile]
\frametitle{Pengolahan String:\newline val}
\begin{block}{val(S, v, e)}
Merupakan prosedur yang mengkonversi suatu string $S$ menjadi data numerik, dan ditampung ke dalam variabel $v$.

Jika terjadi \textbf{error}, variabel $e$ akan berisi nilai yang tidak nol.

Parameter $v$ dan $e$ dipanggil dengan \textit{by reference}.
\end{block}
Contoh:
\begin{lstlisting}
  s := '123';
  nilai := 0;
  e := 0;

  val(s, nilai, e);
  writeln(nilai + 5); (* mencetak 128 *)
\end{lstlisting}

\textbf{Error} bisa terjadi misalnya ketika s bernilai '1a23';
\end{frame}

\begin{frame}[fragile]
\frametitle{Pengolahan String:\newline val (lanj.)}
Parameter e sebenarnya tidak harus ada, sehingga bisa saja ditulis:
\begin{lstlisting}
  s := '123';
  nilai := 0;

  val(s, nilai);
  writeln(nilai + 5); (* mencetak 128 *)
\end{lstlisting}
\end{frame}

\begin{frame}[fragile]
\frametitle{Operasi Tambahan: Konkatenasi}
\begin{itemize}
  \item Konkatenasi adalah penggabungan string.
  \item Pada Pascal, hal ini dapat dilakukan cukup dengan operasi '+', layaknya operasi numerik.
  \item Contoh:
  \begin{lstlisting}
    s := 'Pak';
    t := 'Dengklek';

    gabung := s + t;
    writeln(gabung); (* mencetak 'PakDengklek' *)
  \end{lstlisting}
\end{itemize}
\end{frame}

\begin{frame}[fragile]
\frametitle{Operasi Tambahan: chr dan ord}
\begin{block}{chr(i)}
Merupakan fungsi yang mengembalikan karakter dari kode ASCII i.
\end{block}
\begin{block}{ord(c)}
Merupakan fungsi yang mengembalikan nilai ASCII dari karakter c.
\end{block}
Contoh:
\begin{lstlisting}
  writeln(chr(65));       (* mencetak 'A' *)
  writeln(chr(66));       (* mencetak 'B' *)
  writeln(chr(49));       (* mencetak '1' *)
  writeln(ord('A'));      (* mencetak 65 *)
  writeln(ord('a'));      (* mencetak 97 *)
  writeln(ord('1'));      (* mencetak 49 *)
  writeln(chr(ord('a'))); (* mencetak 'a' *)
\end{lstlisting}
\end{frame}

\begin{frame}[fragile]
\frametitle{Operasi Tambahan: chr dan ord (lanj.)}
\begin{itemize}
  \item Kita dapat mengolah string dengan memproses karakternya satu per satu.
  \item chr dan ord dapat digunakan untuk mengubah karakter suatu string.
  \item Contoh:
  \begin{lstlisting}
    s := 'toki';
    for i := 1 to length(s) do begin
      s[i] := chr(ord(s[i]) - ord('a') + ord('A'));
    end;
    writeln(s);	(* mencetak 'TOKI' *)
  \end{lstlisting}
\end{itemize}
\end{frame}

\section{Wujud Asli String}
\frame{\sectionpage}

\begin{frame}
\frametitle{Fakta Tentang String...}
Sebenarnya, \textbf{string} adalah \textbf{\alert{array of char}}!
\end{frame}

\begin{frame}
\frametitle{Fakta Tentang String... (lanj.)}
\begin{itemize}
  \item Pascal membungkus \textbf{array[0..255] of char} menjadi \textbf{string}, kemudian menambahkan fungsi dan prosedur dasar untuk pengolahan string.
  \item Elemen ke-0 dari string tidak digunakan untuk menampung karakter, melainkan untuk menyimpan \alert{panjang dari string tersebut}.
  \item Artinya, tipe data \textbf{string} hanya bisa menampung maksimal 255 karakter yang menyusunnya.
  \item Gunakan tipe data \textbf{ansistring} untuk menampung karakter yang lebih banyak dari itu, atau buat \textbf{array of char} sendiri $:)$
\end{itemize}
\end{frame}

\begin{frame}[fragile]
\frametitle{Bukti}
\begin{itemize}
  \item Coba jalankan potongan kode berikut!
  \begin{lstlisting}
    s := 'tes';
    writeln(byte(s[0]));
    s := 'tes lagi';
    writeln(byte(s[0]));
  \end{lstlisting}
\end{itemize}
\end{frame}

\begin{frame}
\frametitle{Selanjutnya...}
\begin{itemize}
  \item Pembelajaran kalian tentang Bahasa Pascal sudah cukup untuk bisa menuliskan algoritma-algoritma kompleks.
  \item Berikutnya kita akan mempelajari hal-hal yang lebih berkaitan dengan \textbf{algoritma}, bukan sekedar belajar bahasa.
\end{itemize}
\end{frame}

\end{document}
